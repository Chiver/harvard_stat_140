\documentclass[11pt]{article}
\usepackage[margin=1in]{geometry}
\usepackage{amsmath, amssymb}
\usepackage{graphicx}
\usepackage{booktabs}
\usepackage{hyperref}
\usepackage{parskip}
\usepackage{float}

\title{\textbf{Homework 2}\\Statistics 140}
\date{Due: February 18, 2026}
\author{}

\begin{document}
\maketitle

% =====================================================================
\section*{Problem 1}

\subsection*{(a) Why is the exact Fisher p-value preferable to an approximation?}

The exact Fisher p-value is computed by enumerating \emph{all} possible randomized allocations
and calculating the exact fraction that produce a test statistic at least as extreme as the
observed one.  An approximation based on a smaller random sample of allocations introduces
Monte Carlo noise: the estimated p-value varies across replications and can be above or below
the true value.  Specific drawbacks of the approximation include:

\begin{itemize}
  \item \textbf{Variability.}  The simulated p-value is itself a random variable, so two analysts
    running the same study with different seeds can reach different conclusions.
  \item \textbf{Discreteness near the boundary.}  In a small experiment the exact distribution
    is highly discrete.  A simulation may miss some tie-breaking allocations, distorting the
    p-value around decision thresholds (e.g.\ $\alpha = 0.05$).
  \item \textbf{Misleading precision.}  Quoting a decimal p-value derived from $B$ simulations
    with standard error $\approx\sqrt{p(1-p)/B}$ can convey false precision.
\end{itemize}

When the total number of allocations $\binom{N}{N_T}$ is feasible to enumerate, the exact
computation is straightforward, free of simulation error, and fully reproducible.

\subsection*{(b) Minimum Fisher exact p-value: $N=10$, $N_T=4$, CRE}

In a completely randomized experiment (CRE) with $N=10$ units and $N_T=4$ treated units,
the reference distribution is uniform over all $\binom{10}{4}$ equally-likely allocations.
\[
  \binom{10}{4} = 210.
\]
The minimum possible p-value is attained when exactly one allocation (the observed one) is
as extreme as or more extreme than the observed statistic:
\[
  p_{\min} = \frac{1}{\binom{10}{4}} = \frac{1}{210} \approx 0.00476.
\]

\subsection*{(c) Minimum Fisher exact p-value: \texttt{small\_epi} dataset}

Loading the \texttt{small\_epi.RData} dataset reveals $N=17$ participants with $N_T=10$
exposed to ozone (O\textsubscript{3}) and $N_C=7$ exposed to clean air (CA).

\[
  \binom{17}{10} = 19{,}448.
\]
\[
  p_{\min} = \frac{1}{19{,}448} \approx 5.14 \times 10^{-5}.
\]

\subsection*{(d) Minimum Fisher exact p-value: $N=20$, Bernoulli randomization}

Under Bernoulli randomization each unit is independently assigned to treatment with
probability $1/2$, so there are $2^{20}$ equally-likely allocation vectors (each of
probability $(1/2)^{20}$).
\[
  2^{20} = 1{,}048{,}576.
\]
\[
  p_{\min} = \frac{1}{2^{20}} = \frac{1}{1{,}048{,}576} \approx 9.54 \times 10^{-7}.
\]

\subsection*{(e) Limitations of Figure 2 in Zhong et al.\ (2017) under Bernoulli assignment}

Figure~2 in Zhong et al.\ reports the genome-wide results of an epigenome-wide association
study (EWAS) for $N=10$ volunteers under PM\textsubscript{2.5} exposure.  Assuming a Bernoulli
assignment mechanism, several statistical limitations arise.

\begin{enumerate}
  \item \textbf{Unachievably small p-values.}
    Under Bernoulli randomization the reference distribution contains $2^{10}=1{,}024$
    equiprobable allocations.  The minimum achievable Fisher exact p-value for any single
    test is therefore $1/1{,}024 \approx 0.001$.  Figure~2D (Manhattan plot) and Figure~2B
    (QQ plot) display observed $-\!\log_{10}P$ values reaching $4$--$6$, corresponding to
    p-values of $10^{-4}$--$10^{-6}$---far below the achievable minimum.  Such values
    cannot be validated by exact permutation testing; they rely entirely on asymptotic
    approximations from the linear mixed-effects model.

  \item \textbf{Asymptotic, not exact, inference.}
    The reported p-values come from a parametric model (linear mixed effects with rank-normal
    transformation), not from the Fisher exact test.  With only $N=10$ participants, the
    normality assumptions underlying these asymptotic p-values are difficult to verify.

  \item \textbf{Multiple testing with a limited reference distribution.}
    Approximately 450,000 CpG sites are tested.  Because the Bernoulli reference distribution
    has only $1{,}024$ distinct allocations, the permutation distribution is very coarse.  Any
    Bonferroni- or FDR-corrected threshold that relies on very small p-values cannot be
    reached by exact methods with this sample size.

  \item \textbf{Non-randomized treatment order.}
    The authors explicitly acknowledge that the order of placebo and B-vitamin supplementation
    was \emph{not} randomized (to avoid long washout periods), a direct violation of Bernoulli
    randomization.  Under a Bernoulli mechanism each volunteer should have had an equal
    probability of receiving placebo first or B-vitamin first.  The fixed order introduces a
    potential temporal confound (learning effect, seasonal variation), which the authors attempt
    to address only through covariate adjustment.

  \item \textbf{Small sample limits power.}
    With $N=10$ and a minimum p-value of $\approx 0.001$, the study is underpowered to
    distinguish true signals from noise at stringent genome-wide significance levels.  The
    presentation of the ``top 10 loci'' selected by a combination of effect size and p-value
    (as in Maccani et al.) inflates the apparent significance via winner's curse.
\end{enumerate}

% =====================================================================
\newpage
\section*{Problem 2}

A completely randomized experiment tests whether honey reduces nocturnal cough severity in
6 children ($N_1=3$ treated, $N_0=3$ control).  Observed data:

\begin{center}
\begin{tabular}{ccc}
\toprule
Unit $i$ & $W_i^{\text{obs}}$ & $Y_i^{\text{obs}}$ \\
\midrule
1 & 1 & 3 \\
2 & 1 & 2 \\
3 & 1 & 0 \\
4 & 0 & 4 \\
5 & 0 & 6 \\
6 & 0 & 1 \\
\bottomrule
\end{tabular}
\end{center}

\subsection*{(a) Observed data with potential outcomes}

Under the sharp null hypothesis $H_0: Y_i(1)=Y_i(0)$ for all $i$, both potential outcomes
equal the observed outcome.

\begin{center}
\begin{tabular}{ccccc}
\toprule
Unit $i$ & $W_i^{\text{obs}}$ & $Y_i^{\text{obs}}$ & $Y_i(0)$ & $Y_i(1)$ \\
\midrule
1 & 1 & 3 & 3 & 3 \\
2 & 1 & 2 & 2 & 2 \\
3 & 1 & 0 & 0 & 0 \\
4 & 0 & 4 & 4 & 4 \\
5 & 0 & 6 & 6 & 6 \\
6 & 0 & 1 & 1 & 1 \\
\bottomrule
\end{tabular}
\end{center}

\subsection*{(b) Sharp null hypothesis}

\[
H_0: Y_i(1) = Y_i(0) \quad \text{for all } i = 1, \ldots, 6.
\]
That is, honey has \emph{no effect} on any individual child's cough score.

\subsection*{(c) Matrix of all possible randomized allocations}

With $N=6$ and $N_1=3$, there are $\binom{6}{3}=20$ possible allocations (no duplicates).
Indexing by which units receive treatment ($W_i=1$):

\begin{center}
\small
\begin{tabular}{llr|llr}
\toprule
Treated units & $\mathbf{w}$ & $T$ & Treated units & $\mathbf{w}$ & $T$ \\
\midrule
$\{1,2,3\}$ & (1,1,1,0,0,0) & $-2.000$ & $\{2,3,4\}$ & (0,1,1,1,0,0) & $-1.333$ \\
$\{1,2,4\}$ & (1,1,0,1,0,0) & $0.667$  & $\{2,3,5\}$ & (0,1,1,0,1,0) & $0.000$  \\
$\{1,2,5\}$ & (1,1,0,0,1,0) & $2.000$  & $\{2,3,6\}$ & (0,1,1,0,0,1) & $-3.333$ \\
$\{1,2,6\}$ & (1,1,0,0,0,1) & $-1.333$ & $\{2,4,5\}$ & (0,1,0,1,1,0) & $2.667$  \\
$\{1,3,4\}$ & (1,0,1,1,0,0) & $-0.667$ & $\{2,4,6\}$ & (0,1,0,1,0,1) & $-0.667$ \\
$\{1,3,5\}$ & (1,0,1,0,1,0) & $0.667$  & $\{2,5,6\}$ & (0,1,0,0,1,1) & $0.667$  \\
$\{1,3,6\}$ & (1,0,1,0,0,1) & $-2.667$ & $\{3,4,5\}$ & (0,0,1,1,1,0) & $1.333$  \\
$\{1,4,5\}$ & (1,0,0,1,1,0) & $3.333$  & $\{3,4,6\}$ & (0,0,1,1,0,1) & $-2.000$ \\
$\{1,4,6\}$ & (1,0,0,1,0,1) & $0.000$  & $\{3,5,6\}$ & (0,0,1,0,1,1) & $-0.667$ \\
$\{1,5,6\}$ & (1,0,0,0,1,1) & $1.333$  & $\{4,5,6\}$ & (0,0,0,1,1,1) & $2.000$  \\
\bottomrule
\end{tabular}
\end{center}

\subsection*{(d) Null randomization distribution of $T$}

Under the sharp null, potential outcomes are fixed, so for each allocation $\mathbf{w}$ we
compute
\[
  T(\mathbf{w}) = \frac{1}{3}\sum_{i:\,w_i=1} Y_i^{\text{obs}} - \frac{1}{3}\sum_{i:\,w_i=0} Y_i^{\text{obs}}.
\]
The histogram in Figure~\ref{fig:p2hist} shows all 20 values.

\begin{figure}[H]
  \centering
  \includegraphics[width=0.7\textwidth]{p2_histogram.png}
  \caption{Null randomization distribution of $T$ for the honey experiment (Problem 2).
           The red dashed line marks $T^{\text{obs}} = -2$.}
  \label{fig:p2hist}
\end{figure}

\subsection*{(e) Observed test statistic}

\[
  T^{\text{obs}} = \frac{3+2+0}{3} - \frac{4+6+1}{3} = \frac{5}{3} - \frac{11}{3} = -2.
\]
This value is marked by the red dashed line in Figure~\ref{fig:p2hist}.

\subsection*{(f) Fisher exact p-value}

Because we expect honey to reduce cough (i.e.\ $T^{\text{obs}}<0$), the one-sided p-value
counts allocations with $T\le T^{\text{obs}}=-2$:

\begin{itemize}
  \item $\{1,2,3\}$: $T=-2$ (the observed allocation)
  \item $\{1,3,6\}$: $T=-8/3 \approx -2.667$
  \item $\{2,3,6\}$: $T=-10/3 \approx -3.333$
  \item $\{3,4,6\}$: $T=-2$
\end{itemize}

\[
  p\text{-value} = \frac{4}{20} = 0.20.
\]

There are 4 out of 20 allocations whose test statistic is at least as negative as the
observed value of $-2$.  We do not reject $H_0$ at the 5\% level.

% =====================================================================
\newpage
\section*{Problem 3}

\subsection*{(a) Import \texttt{completelyrandomized.csv}}

The dataset contains 17 participants and 6 CpG sites.  Exposure is coded as
\texttt{exp=2} (ozone) and \texttt{exp=0} (clean air).  Note: the column is labelled
\texttt{cg21036914} in the file (the homework states \texttt{cg21036194}); we use the
actual column name throughout.

\subsection*{(b) Assignment vector \texttt{W.obs}}

\[
  W_i^{\text{obs}} = \begin{cases}1 & \text{if participant } i \text{ was exposed to ozone}\\
  0 & \text{if participant } i \text{ was exposed to clean air.}\end{cases}
\]
There are $N_1=10$ ozone participants and $N_0=7$ clean-air participants.

\subsection*{(c) Matrix $W$ of all possible allocations}

The experiment is a CRE with $N=17$, $N_T=10$.  The number of columns is
\[
  \binom{17}{10} = 19{,}448,
\]
so the matrix $W$ has dimensions $\mathbf{17 \times 19{,}448}$.  Each column is a binary
vector with exactly 10 ones, representing one possible assignment of 10 participants to
ozone.  This equals the total number of ways to choose which 10 of the 17 participants
receive ozone, consistent with a completely randomized design that fixes $N_T=10$.

\subsection*{(d) Outcome vector \texttt{Y.obs} for \texttt{cg00000029}}

\[
\mathbf{Y}^{\text{obs}} = (0.116,\;0.114,\;0.140,\;0.174,\;0.064,\;0.112,\;0.123,\;0.081,\;0.100,\;0.069,\;0.182,\;0.088,\;0.194,\;0.107,\;0.122,\;0.115,\;0.115).
\]

\subsection*{(e) Observed Welch statistic $T^{\text{obs}}$}

\[
  T_{\text{Welch}} = \frac{\bar{Y}_1^{\text{obs}} - \bar{Y}_0^{\text{obs}}}{\sqrt{s_0^2/N_0 + s_1^2/N_1}},
\]
where $s_0^2$ and $s_1^2$ are the sample variances of the clean-air and ozone groups,
respectively.  For \texttt{cg00000029}:
\[
  T^{\text{obs}} = 0.5724248.
\]

\subsection*{(f) Verification with \texttt{t.test}}

Running \texttt{t.test(Y.obs[W.obs==1], Y.obs[W.obs==0])\$statistic} returns
$0.5724248$, confirming that the Welch two-sample $t$-statistic is identical.

\subsection*{(g) Null randomization distribution of $T_{\text{Welch}}$}

For each of the 19,448 allocations, $T_{\text{Welch}}$ is recomputed using the fixed
$\mathbf{Y}^{\text{obs}}$ under the sharp null.  The resulting distribution and the
observed value $T^{\text{obs}}=0.572$ are shown in Figure~\ref{fig:p3g}.

\begin{figure}[H]
  \centering
  \includegraphics[width=0.7\textwidth]{p3g_histogram.png}
  \caption{Null randomization distribution of $T_{\text{Welch}}$ for \texttt{cg00000029}
           (Problem 3g).  Red dashed line marks $T^{\text{obs}}=0.572$.}
  \label{fig:p3g}
\end{figure}

\subsection*{(h) Fisher exact p-value (``extreme'' = ``greater than'')}

\[
  p\text{-value} = \frac{\#\{T_{\text{Welch}} \ge T^{\text{obs}}\}}{19{,}448}
               = \frac{5{,}679}{19{,}448} = 0.2920095.
\]

\subsection*{(i) Comparison to \texttt{t.test} approximation}

The \texttt{t.test} approximating p-value (assuming $T_{\text{Welch}}$ follows a Student's
$t$ distribution under the Neymanian null) is $0.2888397$.  The two values are close but
not identical because:
\begin{itemize}
  \item The Fisher exact p-value is derived from the finite, discrete randomization
    distribution; the $t$-distribution approximation is continuous and asymptotic.
  \item With $N=17$ the approximation is already fairly accurate, but exact and approximate
    values will always differ by a small amount due to the discreteness of the reference
    distribution.
\end{itemize}

\subsection*{(j) Fisher exact p-values for all six CpG sites}

The table below is computed with two nested \texttt{for} loops: the outer loop iterates over
the 6 CpG sites; the inner loop iterates over the 19,448 allocations in $W$.  For the first
three CpG sites, ``extreme'' means $T_{\text{Welch}} \ge T^{\text{obs}}$; for the last three,
``extreme'' means $T_{\text{Welch}} \le T^{\text{obs}}$.

\begin{center}
\begin{tabular}{lrll}
\toprule
CpG site & $T^{\text{obs}}$ & Fisher exact p-value & Approximating p-value \\
\midrule
\texttt{cg00000029} & $\phantom{-}0.5724248$ & $5{,}679/19{,}448 = 0.2920095$ & $0.2888397$ \\
\texttt{cg09008103} & $\phantom{-}7.0847650$ & $1/19{,}448 = 0.0000514$ & $0.0000034$ \\
\texttt{cg14354270} & $\phantom{-}2.7647690$ & $2/19{,}448 = 0.0001028$ & $0.0154687$ \\
\texttt{cg21036914} & $-1.8985240$ & $869/19{,}448 = 0.0446833$ & $0.0385190$ \\
\texttt{cg00673208} & $-5.9502510$ & $1/19{,}448 = 0.0000514$ & $0.0000183$ \\
\texttt{cg20976708} & $-8.1036900$ & $1/19{,}448 = 0.0000514$ & $0.0000004$ \\
\bottomrule
\end{tabular}
\end{center}

\textbf{Discussion in light of Problem 1.}
From Problem 1(c) we know the minimum achievable Fisher exact p-value in this CRE
($N=17$, $N_T=10$) is $1/19{,}448 \approx 5.14\times10^{-5}$.  Four of the six CpG sites
(\texttt{cg09008103}, \texttt{cg14354270}, \texttt{cg00673208}, \texttt{cg20976708})
yield very small exact p-values, with three of them attaining the minimum of $1/19{,}448$.
This means no observed allocation could produce a more extreme statistic than the one
actually observed---the observed allocation is \emph{uniquely} the most extreme.  Despite
corresponding approximating p-values being even smaller (e.g.\ $4\times10^{-7}$ for
\texttt{cg20976708}), we cannot go below the floor $1/19{,}448$ with the exact approach.
This illustrates the trade-off noted in Problem~1(a): the exact p-value is bounded below by
$1/\binom{N}{N_T}$, whereas asymptotic approximations can in principle yield arbitrarily
small values but may not be valid for small $N$.

\subsection*{(k) Six null randomization distributions}

Figure~\ref{fig:p3k} displays the null randomization distributions for all six CpG sites
with the observed statistics marked.

\begin{figure}[H]
  \centering
  \includegraphics[width=\textwidth]{p3k_sixpanel.png}
  \caption{Null randomization distributions of $T_{\text{Welch}}$ for the six CpG sites
           (Problem 3k).  Red dashed lines mark the observed test statistics.}
  \label{fig:p3k}
\end{figure}

\textbf{Comment on shape.}
All six null randomization distributions are approximately bell-shaped and symmetric about
zero, closely resembling a normal (or Student's $t$) distribution.  This is consistent with
the theoretical result that the Welch $t$-statistic converges in distribution to a normal
under the null as $N$ grows.  Even at $N=17$, the continuous nature of the DNA methylation
outcome produces a smooth bell curve.  The centering at zero reflects the fact that under
the sharp null there is no systematic difference between ozone and clean-air groups.

% =====================================================================
\newpage
\section*{Problem 4}

\subsection*{(a) Point estimate and 95\% CI for \texttt{cg00000029}}

The point estimate of the average treatment effect $\tau$ is the difference in sample means:
\[
  \hat\tau = \bar{Y}_1 - \bar{Y}_0.
\]
Using the Neyman/Welch approach, a two-sided 95\% CI is
\[
  \hat\tau \pm t^*_{\nu,0.025} \;\sqrt{s_0^2/N_0 + s_1^2/N_1},
\]
where $\nu$ is the Welch--Satterthwaite degrees of freedom.

For \texttt{cg00000029}: $\hat\tau = 0.009$ and the 95\% CI is $[-0.026,\;0.044]$.  The CI
contains zero, so we cannot reject the null of no mean difference at the 5\% level.  This
is consistent with the relatively large Fisher exact p-value of $0.292$ obtained in
Problem~3.

\subsection*{(b) Verification with \texttt{t.test}}

Running \texttt{t.test(Y.obs[W.obs==1], Y.obs[W.obs==0], alternative="two.sided")} returns
the same $\hat\tau$ and confidence interval.

\subsection*{(c) Results for all six CpG sites}

\begin{center}
\begin{tabular}{lrr}
\toprule
CpG site & $\hat\tau$ & Two-sided 95\% CI \\
\midrule
\texttt{cg00000029} & $\phantom{-}0.009$ & $[-0.026;\;0.044]$ \\
\texttt{cg09008103} & $\phantom{-}0.008$ & $[\phantom{-}0.006;\;0.011]$ \\
\texttt{cg14354270} & $\phantom{-}0.063$ & $[\phantom{-}0.008;\;0.118]$ \\
\texttt{cg21036914} & $-0.006$ & $[-0.013;\;0.001]$ \\
\texttt{cg00673208} & $-0.007$ & $[-0.010;\;-0.004]$ \\
\texttt{cg20976708} & $-0.030$ & $[-0.037;\;-0.022]$ \\
\bottomrule
\end{tabular}
\end{center}

\textbf{Interpretation.}
\begin{itemize}
  \item \texttt{cg00000029}: CI contains 0; no significant effect of ozone.
  \item \texttt{cg09008103} and \texttt{cg14354270}: positive $\hat\tau$ with CIs entirely
    above zero, indicating ozone increases methylation at these sites.
  \item \texttt{cg21036914}: CI barely contains 0, marginal evidence for a decrease.
  \item \texttt{cg00673208} and \texttt{cg20976708}: negative $\hat\tau$ with CIs entirely
    below zero, indicating ozone \emph{decreases} methylation at these sites.
\end{itemize}
These results are consistent with the p-values from Problem~3: the sites with the smallest
Fisher exact p-values ($1/19{,}448$) also have CIs that exclude zero.

\end{document}
